\documentclass[12pt,a4paper]{article}
\usepackage[utf8]{inputenc}
\usepackage[russian]{babel}
\usepackage[left=7mm,right=7mm,top=10mm,bottom=10mm,bindingoffset=0mm]{geometry}
\usepackage{graphicx, nicefrac}
\usepackage{amsmath}
\usepackage{amsfonts}
\usepackage{amssymb}
\usepackage{stackengine}

\geometry{papersize={10.4cm, 110cm}}

\newcommand\xrowht[2][0]{\addstackgap[.5\dimexpr#2\relax]{\vphantom{#1}}}


\graphicspath{}
\DeclareGraphicsExtensions{.pdf,.png,.jpg}


\newcommand{\ds}{\displaystyle}

\newcommand{\im}{\mathrm{Im}}
\newcommand{\re}{\mathrm{Re}}
\renewcommand{\arg}{\mathrm{Arg}}

\newcommand{\ro}{\hat{\rho}}
\newcommand{\z}{\hat{Z}}
\newcommand{\y}{\hat{Y}}
\newcommand{\zv}{Z_{\text{В}}}
\newcommand{\yv}{Y_{\text{В}}}
\newcommand{\zn}{\hat{Z}_{\text{H}}}
\newcommand{\yn}{\hat{Y}_{\text{H}}}
\newcommand{\zz}{\hat{Z}_0}
\newcommand{\kb}{K_{\text{БВ}}}
\newcommand{\ks}{K_{\text{СВН}}}
\newcommand{\dxl}{\displaystyle\frac{\Delta x}{\lambda}}
\newcommand{\um}{\hat{U}^-}
\newcommand{\up}{\hat{U}^+}
\newcommand{\jm}{\hat{I}^-}
\newcommand{\jp}{\hat{I}^+}


\begin{document}
	\thispagestyle{empty}
	\textbf{Управление:}
	\begin{itemize}
		\item Движение мыши с зажатой левой кнопкой --- построение окружностей постоянного активного $R$ и
		реактивного $X$ сопротивлений, проходящих через данную точку, лежащую на окружности радиуса
		равного модулю коэффициента отражения $\rho$.
		\item Движение мыши с зажатой правой кнопкой --- смещение относительно точки $\ro = 0$.
		\item Колёсико мыши --- изменение масштаба с шагом 25\%.
		\item SHIFT --- фиксирует значение по $y$ т.е. $\im\,\ro$\,.
		\item CTRL --- фиксирует значение по $x$ т.е. $\re\,\ro$\,.
		\item ALT --- фиксирует значение на окружности равного $\kb$($\ks$) т.е. $\rho$\,.
	\end{itemize}
	
	\textbf{Задачи:}
	\begin{itemize}
		\item[1)] Определение $\z=R+jX$ и $\y=G+jB$ по заданным: $\rho$ или $\kb$ или $\ks$ и $\dxl$;
		\item[2)] Определение $\ro$, $\kb$, $\ks$ и $\dxl$ по заданному $\z=R+jX$ или $\y=G+jB$;
	\end{itemize}
	
	Рассчёт значений искомых величин происходит по нажатию \texttt{<Enter>} в поле ввода задаваемого значения.
	
	При вводе ненормированных значений $R$, $X$, $G$ и $B$ необходимо ввести значение $\zv$ или
	$\yv$ в соответствущее поле.
	
	Формат вывода величины $\dxl$ :
	\begin{center}
		$\boxed{\zn\mathstrut} \rightarrow \boxed{\,\sim\mathstrut}$
		$\biggl( \boxed{\,\sim\mathstrut} \rightarrow \boxed{\zn\mathstrut} \biggr)$
	\end{center}
	Вводится только значение <<к генератору>>. \\
	
	\textbf{Вывод уравнений для вычисления параметров линии и построения соответствующих кривых для
	произвольной точки диаграмы.} \\
	\begin{equation}
		\begin{cases}
			\hat{U} = \up + \um; \\
			\hat{I} = \jp + \jm;
		\end{cases}
	\end{equation}
	Импеданс нагрузки:
	\begin{equation}
		\ds \zn = \frac1{\yn} = \frac{\hat{U}}{\hat{I}} = \frac{\up + \um}{\jp + \jm};
	\end{equation}
	Коэффициент отражения:
	\begin{equation}
		\ds \ro = \frac{\hat{U}^-}{\hat{U}^+};
	\end{equation} 
	Тогда импеданс нагрузки:
	\begin{equation}
		\ds \zn = \frac{\up(1+\ro)}{\jp(1-\ro)} = \zv \frac{1+\ro}{1-\ro};
	\end{equation} 
	Выразим коэффициент отражения:
	\begin{equation}
		\ro = \frac{\zn - \zv}{\zn + \zv};
	\end{equation} 
	Введем нормированный импеданс:
	\begin{equation}
		\zz = \frac{\zn}{\zv};
	\end{equation} 
	Тогда коэффициент отражения:
	\begin{equation}
		\ds \ro = \frac{\zz - 1}{\zz + 1} = \frac{R+jX-1}{R+jX+1};
	\end{equation}
	\begin{center}
		$\ds \ro = \frac{\left( R+jX-1 \right)\left( R-jX+1 \right)}{\left( R+jX+1 \right)
		\left( R-jX+1 \right)}$;
	\end{center}
	\begin{center}
		$\ds \ro = x+jy = \frac{R^2+X^2-1+2jX}{{(R+1)}^2+X^2}$;
	\end{center}
	\begin{center}
		$\begin{cases}
			\ds \re\,\ro = x = \frac{R^2+X^2-1}{{(R+1)}^2+X^2}; \\[12pt]
			\ds \im\,\ro = y = \frac{2X}{{(R+1)}^2+X^2};
		\end{cases}$
	\end{center}
	Выражая $X^2$ через $\re\,\ro$ :
	\begin{center}
		$\ds X^2 = \frac{x(R+1)^2-R^2+1}{1-x};$
	\end{center}
	и подставляя в $\im\,\ro$, получим :
	\begin{equation}
		\ds {\left(x-\frac{R}{R+1}\right)}^2 + {y}^2 = {\left(\frac{1}{R+1}\right)}^2;
	\end{equation}
	Полученное уравнение --- уравнение окружности:
	\begin{center}
		$\begin{cases}
			\ds \left( \frac{R}{R+1}; 0 \right) & \text{\quad --- \quad центр}; \\[12pt]
			\ds \hspace{15pt} \frac1{R+1} & \text{\quad --- \quad радиус};
		\end{cases}$
	\end{center}
	В параметрическом виде:
	\begin{equation}
		\begin{cases}
			\ds x(t) = \frac{R}{R+1} + \frac1{R+1} \cos(t); \\[12pt]
			\ds y(t) = \frac1{R+1} \sin(t); \\[12pt]
			0 \leq t \leq 2\pi;
		\end{cases}
	\end{equation}
	Выражая $R$ через $\im\,\ro$ :
	\begin{center}
		$\ds R = \frac{\pm\sqrt{-yX(yX-2)}-y}{y};$
	\end{center}
	и подставляя в $\re\,\ro$, получим :
	\begin{equation}
		\ds {(x-1)}^2+{\left(y-\frac1X\right)}^2 = {\left(\frac1X\right)}^2;
	\end{equation}
	Полученное уравнение --- уравнение окружности:
	\begin{center}
		$\begin{cases}
			\ds \left( 1; \frac1X \right) & \text{\quad --- \quad центр}; \\[12pt]
			\ds \hspace{14pt} \frac1X & \text{\quad --- \quad радиус};
		\end{cases}$
	\end{center}
	В параметрическом виде:
	\begin{equation}
		\begin{cases}
			\ds x(t) = 1 + \frac1X \cos(t); \\[12pt]
			\ds y(t) = \frac1X + \frac1X \sin(t); \\[12pt]
			\ds \frac{\pi}2 \leq t \leq 2 \arctg \left( \frac1X \right);
		\end{cases}
	\end{equation}
	
	Окружности постоянного активного сопротивления строятся по уравнениям (9), а
	дуги постоянного реактивного сопротивления по уравнениям (11). \\
	
	Преобразовывая (8) и (10) получим :
	\begin{equation}
		\begin{cases}
			\ds x^2 + y^2 - 2x\frac{R}{R+1} + \frac{R^2+1}{{(R+1)}^2} = 0; \\[12pt]
			\ds x^2 + y^2 - 2x - 2y\frac1X + 1 = 0;
		\end{cases}
	\end{equation}
	
	Введём $k = x^2 + y^2 = \rho^2$ \quad --- \quad имеет смысл определителя (показывает принадлежит
	ли точка с координатами $x$ и $y$ области допустимых значений $\rho$).
	
	Уравнения для нахождения $R$ и $X$ по заданным $x$ и $y$ можно выразить из (12):
	\begin{equation}
		\begin{cases}
			\ds R = \frac{1-k}{1+k-2x}; \\[12pt]
			\ds X = \frac{2y}{1+k-2x}; \\
		\end{cases}
	\end{equation}
	
	Модуль коэффициента отражения через $x$ и $y$ выражается как:
	\begin{equation}
		\rho = \sqrt{x^2 + y^2} = \sqrt{k};
	\end{equation}
	
	Аргумент коэффициента отражения через $x$ и $y$ выражается как:
	\begin{equation}
		\ds \arg\,\ro = \arctg \left( \frac{y}{x} \right);
	\end{equation}
	
	Коэффициенты бегущей и стоячей волн находтся через $\rho$:
	\begin{equation}
		\begin{cases}
			\ds \kb = \frac{1-\rho}{1+\rho}; \\[12pt]
			\ds \ks = \frac{1+\rho}{1-\rho}; \\[12pt]
		\end{cases}
	\end{equation}
	
	Величина $\dxl$ линейно зависит от аргумента коэффициента отражения:
	\begin{equation}
		\dxl = \frac14 \left[ 1 - \frac{\arg\,\ro}{\pi} \right];
	\end{equation}
	
	Таким образом для произвольной точки $(x;y)$ из области допустимых значений $\rho$ можно вычислить:
	\begin{center}
		$R, \quad X, \quad \zz, \quad \ro, \quad \kb, \quad \ks, \quad \dxl.$
	\end{center}
	
	\textbf{Решение задачи I типа.} \\
	$R$ и $X$ находятся по формулам (13), где:
	\begin{equation}
		\begin{cases}
			k = \rho^2; \\[5pt]
			x = \re\,\ro = \rho \cos(\varphi); \\[5pt]
			y = \im\,\ro = \rho \sin(\varphi); \\
		\end{cases}
	\end{equation}
	
	Модуль коэффициента отражения $\rho$ либо задан непосредственно, либо вычисляется по формулам:
	\begin{equation}
		\ds \rho = \frac{1-\kb}{1-\kb} = \frac{\ks-1}{\ks+1};
	\end{equation}
	
	Аргумент коэффициента отражения $\varphi$ вычисляется по формуле:
	\begin{equation}
		\varphi = \arg\,\ro = \pi \left[ 1 - 4\dxl \right];
	\end{equation}
	
	\textbf{Решение задачи II типа.} \\
	Действительная и мнимая части коэффициента отражения вычисляются через заданные $R$ и $X$
	по формуле (7).
	
	Модуль и аргумент коэффициента отражения находятся через действительную и мнимую части.
	
	Коэффициенты бегущей и стоячей волн находятся по формуле (16). Величина $\dxl$ по формуле (17).
\end{document}
